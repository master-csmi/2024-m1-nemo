\documentclass{beamer}
\usepackage{graphicx}

\title{Squirmers: Understanding and Simulating their interactions}
\author{Robin and Justine}

\begin{document}
\begin{frame}
    \titlepage
\end{frame}

\begin{frame}{What is a Squirmer ?}
    \begin{columns}[T]
        \begin{column}{0.5\textwidth}
            \begin{itemize}
                \item Introduced by James Lighthill in 1952
                \item Refined by John Blake in 1971
                \item Model for a spherical microswimmer
                \item Cannot modelize cilia so we impose boundary conditions
            \end{itemize}
        \end{column}
        \begin{column}{0.5\textwidth}
            \centering
            \includegraphics[width=\textwidth]{images/squirmer.png}
        \end{column}
    \end{columns}
\end{frame}

\begin{frame}{Objectives}
    \begin{itemize}
        \item Understand the "Squirmers" model
        \item Study the dynamics between two squirmers
        \begin{itemize}
            \item Study the interactions between two squirmers
            \item Reformulate the formulas in our own terms
            \item Develop code to calculate the movement of two squirmers
        \end{itemize}
        \item Numerical Experiments
        \begin{itemize}
            \item Verify if our results align with previous studies
            \item Verify if changing the $\beta$ and distance parameters affects behavior
        \end{itemize}
    \end{itemize}
\end{frame}

\begin{frame}{Roadmap}
    
\end{frame}

\begin{frame}{Bibliography}
    \begin{thebibliography}{3}
        \bibitem{Brumley} D.R. Brumley and T.J. Pedley, \emph{Stability of arrays of bottom-heavy spherical squirmers}, American Physical Society, 2019
        \bibitem{Lauga} Théry A., Maaß C.C. and Lauga E., \emph{Hydrodynamic interactions between squirmers near walls: far-field dynamics and near-field cluster stability}, Royal Society Open Science, 2023
        \bibitem{Wikipedia} Wikipédia, \emph{Squirmer}, Wikipédia, 2022
    \end{thebibliography}
\end{frame}

\end{document}