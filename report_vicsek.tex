\documentclass{article}
\usepackage[]{url}
\usepackage{graphicx}
\usepackage{hyperref}
\usepackage{mathtools}
\usepackage{tikz}
\usepackage{subcaption}

\begin{document}
\title{Collective motion of squirmers in confined environments}
\author{Roth Robin\\
\\
Supervisors: Van Landeghem Céline, Giraldi Laetitia,\\ Agathe Chouippe}
\date{May, 2024}
\maketitle

\tableofcontents

\section{Context}
The collective behavior of active particles is well studied in the literature. 
The Vicsek model is frequently used to model these active particles. 
The aim of this internship is to perform a similar study considering active rigid swimmers, known as squirmers, 
in various confined domains and to compare the collective motion of the squirmers with that of the active particles using the Vicsek model. 
The motion of the squirmers is simulated using two different models: a continuous model that approximates hydrodynamic and 
steric forces, and a full model using the finite element library Feel++.

\section{Objectives}
The main goal is to model the dynamics of a schoal of squirmers within confined environments, 


\end{document}